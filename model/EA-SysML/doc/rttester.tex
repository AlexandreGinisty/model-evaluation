\subsubsection{Tool overview}

The RT-Tester test automation tool, made by Verified
\cite{verified_website}, performs automatic test generation, test
execution and real-time test evaluation.  It supports different
testing approach such as unit testing, software integration testing
for component, hardware/software integration testing and system
integration testing.  The RT-Tester version we had used, follows the
model-based testing approach \cite{Peleska2011,rttmbtreport2011} and
it provides the following features :
\begin{itemize}
\item Automated Test Case Generation 
\item Automated Test Data Generation 
\item Automated Test Procedure Generation 
\item Automated Requirement Tracing 
\item Test Management system 
\end{itemize}
Starting from a test model design with UML/SYML, the RT-tester fully
automatically generates test cases. They are then specified as test
data (sequences of stimuli with timing constraints) and used to
stimulate the SUT and run concurently with the generated test
oracles. The test procedure is the combination of the test oracles and
the SUT that can be compiled and executed.

The tool supports test cases/data generation for structural
testing. It automatically generates  reach statement coverage, branch coverage and
modified condition/decision coverage (MC/DC) as far as this is possible.
The test cases may all be linked to requirements ensuring a complete
requirement traceability. 
Additionally RT-tester may produce test cases/data from a LTL
formula, since a LTL formula describes a possible run of the model.


Finally the tool may produce the documentation of tests for
certification purposes. For each test cases the following document are
produced :
\begin{itemize}
\item {\em Test procedure}: that specifies  how one test case can be
  executed, its associated test data produced and how the SUT
  reactions are evaluated against the expected results.
\item {\em Test report}: that summarizes all relevant information
  about the test execution.
\end{itemize}

In \cite{brauer_efficient_2012}, a general approach  on how to qualify
model-based testing tool according to the standard ISO 26262 ad RTCA
DO178C has been proposed and applied with success to the RT-tester
tool. Following the same  approach compatibility with the CENELEC EN50128
may be easily done. 




\subsubsection{Test generation}
RT-Tester is able to automatically generate test cases from the test
model.
The first step may be to generate the test cases that cover all the
model artifacts : a test to reach all transitions, control state and
perform MC/DC coverage of the model. The test cases are
related to their attached requirement. This is automatically derived
from the requirement diagram. 

The test campaign may then be guided by requirement coverage, one
can select the test cases associated to a requirement, or
by test cases coverage, one select the tests to be covered.

Finally, it is possible to add user define test cases. This has been
done to import test cases part from the SUBSET-076. his has been made
manually by translating the test cases description from the word
document as an LTL formula. 
For example the test case 1 of feature  177, tests the requirement :
"Establishing a radio communication initiated by RBC''.  It checks
that after receiving a connection indication of the safe radio
from the RTM, the OBU  shall confirm the establishment and shall
consider the session established following the process describes in
the requirements 3.5.3. This has been translated into the following
formula:
\begin{multicols}{2}
\begin{verbatim}
// Initial state NO_COM
[
setUp == 0 &&
radioComSesssionEstablished == 0 && 
M_Level >= 3 && 
M_Mode != 3 &&
M_Mode != 4 &&
M_Mode != 5 &&
M_Mode != 9 &&
M_Mode != 10 &&
radioHole == NONE &&
orderOnBoard == NONE  &&
messageIn == NO_MESSAGE
] && 
[
setUp == 0 &&
radioComSesssionEstablished == 0 && 
M_Level >= 3 && 
M_Mode != 3 &&
M_Mode != 4 &&
M_Mode != 5 &&
M_Mode != 9 &&
M_Mode != 10 &&
radioHole == NONE &&
orderOnBoard == NONE  &&
messageIn == NO_MESSAGE 
]  
Until
(// Safe radio set up
[
  setUp == 1 &&
  IMR.messageFromRBC == 38 
] &&
Finally 
// Message session established send 
// session considered as established
[
  IMR.messageFromRBC == 38 &&
  IMR.messageToRBC == 159 &&
  IMR.radioComSesssionEstablished == 1
])
\end{verbatim}
\end{multicols}
\begin{table}[htbp]
\centering
\begin{tabular}{lr}\toprule
  Test cases covered & \# tests generated  \\\midrule
  Basic state & 11 \\
  Transition & 28 \\
  MC/DC & 63 \\
  LTL & 4 \\
  Test 177 & 2 \\ \midrule
  Total Tests& 108\\\bottomrule
\end{tabular}

\raggedleft Total Requirement cover 36
\caption{\label{tbl:test_summary} Test cases generation summary}
\end{table}

Table \ref{tbl:test_summary} summarizes the set of test cases
generated and the related requirements. The line LTL represents the
requirement that has been rewrite with a LTL formula. The test 177
represents the translation as LTL formula of the two test cases described
in SUBSET 076. This set of test cases covers 36 requirements from SUBSET-026-chap 3.5.

%%% Local Variables: 
%%% mode: latex
%%% TeX-master: "descriptionReport"
%%% End: 

